% Options for packages loaded elsewhere
\PassOptionsToPackage{unicode}{hyperref}
\PassOptionsToPackage{hyphens}{url}
%
\documentclass[
]{article}
\usepackage{lmodern}
\usepackage{amssymb,amsmath}
\usepackage{ifxetex,ifluatex}
\ifnum 0\ifxetex 1\fi\ifluatex 1\fi=0 % if pdftex
  \usepackage[T1]{fontenc}
  \usepackage[utf8]{inputenc}
  \usepackage{textcomp} % provide euro and other symbols
\else % if luatex or xetex
  \usepackage{unicode-math}
  \defaultfontfeatures{Scale=MatchLowercase}
  \defaultfontfeatures[\rmfamily]{Ligatures=TeX,Scale=1}
\fi
% Use upquote if available, for straight quotes in verbatim environments
\IfFileExists{upquote.sty}{\usepackage{upquote}}{}
\IfFileExists{microtype.sty}{% use microtype if available
  \usepackage[]{microtype}
  \UseMicrotypeSet[protrusion]{basicmath} % disable protrusion for tt fonts
}{}
\makeatletter
\@ifundefined{KOMAClassName}{% if non-KOMA class
  \IfFileExists{parskip.sty}{%
    \usepackage{parskip}
  }{% else
    \setlength{\parindent}{0pt}
    \setlength{\parskip}{6pt plus 2pt minus 1pt}}
}{% if KOMA class
  \KOMAoptions{parskip=half}}
\makeatother
\usepackage{xcolor}
\IfFileExists{xurl.sty}{\usepackage{xurl}}{} % add URL line breaks if available
\IfFileExists{bookmark.sty}{\usepackage{bookmark}}{\usepackage{hyperref}}
\hypersetup{
  pdftitle={Week 03 Assignment/ Week 04 Resubmission},
  pdfauthor={Jessica Smith},
  hidelinks,
  pdfcreator={LaTeX via pandoc}}
\urlstyle{same} % disable monospaced font for URLs
\usepackage[margin=1in]{geometry}
\usepackage{color}
\usepackage{fancyvrb}
\newcommand{\VerbBar}{|}
\newcommand{\VERB}{\Verb[commandchars=\\\{\}]}
\DefineVerbatimEnvironment{Highlighting}{Verbatim}{commandchars=\\\{\}}
% Add ',fontsize=\small' for more characters per line
\usepackage{framed}
\definecolor{shadecolor}{RGB}{248,248,248}
\newenvironment{Shaded}{\begin{snugshade}}{\end{snugshade}}
\newcommand{\AlertTok}[1]{\textcolor[rgb]{0.94,0.16,0.16}{#1}}
\newcommand{\AnnotationTok}[1]{\textcolor[rgb]{0.56,0.35,0.01}{\textbf{\textit{#1}}}}
\newcommand{\AttributeTok}[1]{\textcolor[rgb]{0.77,0.63,0.00}{#1}}
\newcommand{\BaseNTok}[1]{\textcolor[rgb]{0.00,0.00,0.81}{#1}}
\newcommand{\BuiltInTok}[1]{#1}
\newcommand{\CharTok}[1]{\textcolor[rgb]{0.31,0.60,0.02}{#1}}
\newcommand{\CommentTok}[1]{\textcolor[rgb]{0.56,0.35,0.01}{\textit{#1}}}
\newcommand{\CommentVarTok}[1]{\textcolor[rgb]{0.56,0.35,0.01}{\textbf{\textit{#1}}}}
\newcommand{\ConstantTok}[1]{\textcolor[rgb]{0.00,0.00,0.00}{#1}}
\newcommand{\ControlFlowTok}[1]{\textcolor[rgb]{0.13,0.29,0.53}{\textbf{#1}}}
\newcommand{\DataTypeTok}[1]{\textcolor[rgb]{0.13,0.29,0.53}{#1}}
\newcommand{\DecValTok}[1]{\textcolor[rgb]{0.00,0.00,0.81}{#1}}
\newcommand{\DocumentationTok}[1]{\textcolor[rgb]{0.56,0.35,0.01}{\textbf{\textit{#1}}}}
\newcommand{\ErrorTok}[1]{\textcolor[rgb]{0.64,0.00,0.00}{\textbf{#1}}}
\newcommand{\ExtensionTok}[1]{#1}
\newcommand{\FloatTok}[1]{\textcolor[rgb]{0.00,0.00,0.81}{#1}}
\newcommand{\FunctionTok}[1]{\textcolor[rgb]{0.00,0.00,0.00}{#1}}
\newcommand{\ImportTok}[1]{#1}
\newcommand{\InformationTok}[1]{\textcolor[rgb]{0.56,0.35,0.01}{\textbf{\textit{#1}}}}
\newcommand{\KeywordTok}[1]{\textcolor[rgb]{0.13,0.29,0.53}{\textbf{#1}}}
\newcommand{\NormalTok}[1]{#1}
\newcommand{\OperatorTok}[1]{\textcolor[rgb]{0.81,0.36,0.00}{\textbf{#1}}}
\newcommand{\OtherTok}[1]{\textcolor[rgb]{0.56,0.35,0.01}{#1}}
\newcommand{\PreprocessorTok}[1]{\textcolor[rgb]{0.56,0.35,0.01}{\textit{#1}}}
\newcommand{\RegionMarkerTok}[1]{#1}
\newcommand{\SpecialCharTok}[1]{\textcolor[rgb]{0.00,0.00,0.00}{#1}}
\newcommand{\SpecialStringTok}[1]{\textcolor[rgb]{0.31,0.60,0.02}{#1}}
\newcommand{\StringTok}[1]{\textcolor[rgb]{0.31,0.60,0.02}{#1}}
\newcommand{\VariableTok}[1]{\textcolor[rgb]{0.00,0.00,0.00}{#1}}
\newcommand{\VerbatimStringTok}[1]{\textcolor[rgb]{0.31,0.60,0.02}{#1}}
\newcommand{\WarningTok}[1]{\textcolor[rgb]{0.56,0.35,0.01}{\textbf{\textit{#1}}}}
\usepackage{graphicx,grffile}
\makeatletter
\def\maxwidth{\ifdim\Gin@nat@width>\linewidth\linewidth\else\Gin@nat@width\fi}
\def\maxheight{\ifdim\Gin@nat@height>\textheight\textheight\else\Gin@nat@height\fi}
\makeatother
% Scale images if necessary, so that they will not overflow the page
% margins by default, and it is still possible to overwrite the defaults
% using explicit options in \includegraphics[width, height, ...]{}
\setkeys{Gin}{width=\maxwidth,height=\maxheight,keepaspectratio}
% Set default figure placement to htbp
\makeatletter
\def\fps@figure{htbp}
\makeatother
\setlength{\emergencystretch}{3em} % prevent overfull lines
\providecommand{\tightlist}{%
  \setlength{\itemsep}{0pt}\setlength{\parskip}{0pt}}
\setcounter{secnumdepth}{-\maxdimen} % remove section numbering

\title{Week 03 Assignment/ Week 04 Resubmission}
\author{Jessica Smith}
\date{}

\begin{document}
\maketitle

\begin{Shaded}
\begin{Highlighting}[]
\KeywordTok{library}\NormalTok{(devtools);}
\NormalTok{my.source =}\StringTok{ 'local'}\NormalTok{;}
\NormalTok{local.path =}\StringTok{ "C:}\CharTok{\textbackslash{}\textbackslash{}}\StringTok{Users}\CharTok{\textbackslash{}\textbackslash{}}\StringTok{jsmit}\CharTok{\textbackslash{}\textbackslash{}}\StringTok{Desktop}\CharTok{\textbackslash{}\textbackslash{}}\StringTok{WSU}\CharTok{\textbackslash{}\textbackslash{}}\StringTok{DataAnalytics}\CharTok{\textbackslash{}\textbackslash{}}\StringTok{STAT419}\CharTok{\textbackslash{}\textbackslash{}}\StringTok{WSU_STATS419_FALL2020"}\NormalTok{;}
\NormalTok{local.data.path =}\StringTok{ "R:/WSU_STATS419_FALL2020/"}\NormalTok{;}
\CommentTok{#setwd(local.path)}
\NormalTok{knitr}\OperatorTok{::}\NormalTok{opts_knit}\OperatorTok{$}\KeywordTok{set}\NormalTok{(}\DataTypeTok{root.dir =}\NormalTok{ local.path)}
\KeywordTok{source}\NormalTok{( }\KeywordTok{paste0}\NormalTok{(local.path,}\StringTok{"}\CharTok{\textbackslash{}\textbackslash{}}\StringTok{functions}\CharTok{\textbackslash{}\textbackslash{}}\StringTok{libraries.R"}\NormalTok{), }\DataTypeTok{local=}\NormalTok{T );}

\CommentTok{#install_github("MonteShaffer/humanVerseWSU/humanVerseWSU");}
\KeywordTok{library}\NormalTok{(humanVerseWSU); }\CommentTok{# if your functions have the same name as the humanVerseWSU functions, there may be a collision ...  order of sourcing/library (which comes first, second, third, etc.) matters.}
\end{Highlighting}
\end{Shaded}

\hypertarget{matrix}{%
\section{Matrix}\label{matrix}}

Create the ``rotate matrix'' functions as described in lectures. Apply
to the example ``myMatrix''.

\begin{Shaded}
\begin{Highlighting}[]
\KeywordTok{source}\NormalTok{( }\KeywordTok{paste0}\NormalTok{(local.path,}\StringTok{"/functions/functions-matrix.R"}\NormalTok{), }\DataTypeTok{local=}\NormalTok{T );}
\CommentTok{#install_github("/jsmith0434/WSU_STATS419_FALL2020/functions")}

\NormalTok{myMatrix =}\StringTok{ }\KeywordTok{matrix}\NormalTok{ ( }\KeywordTok{c}\NormalTok{ (}
                                            \DecValTok{1}\NormalTok{, }\DecValTok{0}\NormalTok{, }\DecValTok{2}\NormalTok{,}
                                            \DecValTok{0}\NormalTok{, }\DecValTok{3}\NormalTok{, }\DecValTok{0}\NormalTok{,}
                                            \DecValTok{4}\NormalTok{, }\DecValTok{0}\NormalTok{, }\DecValTok{5}
\NormalTok{                                            ), }\DataTypeTok{nrow=}\DecValTok{3}\NormalTok{, }\DataTypeTok{byrow=}\NormalTok{T);}

\CommentTok{# dput(myMatrix); # useful}
\end{Highlighting}
\end{Shaded}

\begin{Shaded}
\begin{Highlighting}[]
\NormalTok{humanVerseWSU}\OperatorTok{::}\KeywordTok{transposeMatrix}\NormalTok{(myMatrix);}
\end{Highlighting}
\end{Shaded}

\begin{verbatim}
##      [,1] [,2] [,3]
## [1,]    1    0    4
## [2,]    0    3    0
## [3,]    2    0    5
\end{verbatim}

\begin{Shaded}
\begin{Highlighting}[]
\KeywordTok{rotateMatrix90}\NormalTok{(myMatrix);  }\CommentTok{# clockwise ... }
\end{Highlighting}
\end{Shaded}

\begin{verbatim}
##      [,1] [,2] [,3]
## [1,]    4    0    1
## [2,]    0    3    0
## [3,]    5    0    2
\end{verbatim}

\begin{Shaded}
\begin{Highlighting}[]
\KeywordTok{rotateMatrix180}\NormalTok{(myMatrix);}
\end{Highlighting}
\end{Shaded}

\begin{verbatim}
##      [,1] [,2] [,3]
## [1,]    5    0    4
## [2,]    0    3    0
## [3,]    2    0    1
\end{verbatim}

\begin{Shaded}
\begin{Highlighting}[]
\KeywordTok{rotateMatrix270}\NormalTok{(myMatrix);}
\end{Highlighting}
\end{Shaded}

\begin{verbatim}
##      [,1] [,2] [,3]
## [1,]    2    0    5
## [2,]    0    3    0
## [3,]    1    0    4
\end{verbatim}

\begin{Shaded}
\begin{Highlighting}[]
\CommentTok{# rotateMatrix(mat,a) }\AlertTok{###}\CommentTok{ one function using a switch statement ...}
\end{Highlighting}
\end{Shaded}

\hypertarget{iris}{%
\section{IRIS}\label{iris}}

Recreate the graphic for the IRIS Data Set using R. Same titles, same
scales, same colors. See:
\url{https://en.wikipedia.org/wiki/Iris_flower_data_set#/media/File:Iris_dataset_scatterplot.svg}

\begin{Shaded}
\begin{Highlighting}[]
\KeywordTok{data}\NormalTok{(iris)}
\KeywordTok{pairs}\NormalTok{(iris[}\DecValTok{1}\OperatorTok{:}\DecValTok{4}\NormalTok{], }\DataTypeTok{main =} \StringTok{"Iris Data (red=setosa, green=versicolor,blue-virginica)"}\NormalTok{, }
      \DataTypeTok{pch =} \DecValTok{21}\NormalTok{, }\DataTypeTok{bg =} \KeywordTok{c}\NormalTok{(}\StringTok{"red"}\NormalTok{, }\StringTok{"green3"}\NormalTok{, }\StringTok{"blue"}\NormalTok{)[}\KeywordTok{unclass}\NormalTok{(iris}\OperatorTok{$}\NormalTok{Species)])}
\end{Highlighting}
\end{Shaded}

\includegraphics{04_datasets_notebook_resubmission_files/figure-latex/mychunk-iris-1.pdf}

Sentences: {[}Right 2-3 sentences concisely defining the IRIS Data Set.
Maybe search KAGGLE for a nice template. Be certain the final writeup
are your own sentences (make certain you modify what you find, make it
your own, but also cite where you got your ideas from). NOTE: Watch the
video, Figure 8 has a +5 EASTER EGG.{]}

The well known iris data set contains 50 measurements, each from 3
species of iris flower. The metrics include petal length, petal width,
sepal length, and sepal width. The dataset is commonly used to teach
clustering analysis and to demonstrate basic programmatic functionality.
(Kaggle, 2020)

\hypertarget{personality}{%
\section{Personality}\label{personality}}

\hypertarget{cleanup-raw}{%
\subsection{Cleanup RAW}\label{cleanup-raw}}

Import ``personality-raw.txt'' into R. Remove the V00 column. Create two
new columns from the current column ``date\_test'': year and week. Stack
Overflow may help:
\url{https://stackoverflow.com/questions/22439540/how-to-get-week-numbers-from-dates}
\ldots{} Sort the new data frame by YEAR, WEEK so the newest tests are
first \ldots{} The newest tests (e.g., 2020 or 2019) are at the top of
the data frame. Then remove duplicates using the unique function based
on the column ``md5\_email''. Save the data frame in the same
``pipe-delimited format'' ( \textbar{} is a pipe ) with the headers. You
will keep the new data frame as ``personality-clean.txt'' for future
work (you will not upload it at this time). In the homework, for this
tasks, report how many records your raw dataset had and how many records
your clean dataset has.

\begin{Shaded}
\begin{Highlighting}[]
\CommentTok{# working directory is good-to-go}

\NormalTok{myFile =}\StringTok{ }\KeywordTok{paste0}\NormalTok{(local.path,}\StringTok{"/datasets/personality/personality-raw.txt"}\NormalTok{);}

\CommentTok{#read in the data}
\NormalTok{personality =}\StringTok{ }\KeywordTok{read.table}\NormalTok{(myFile, }\DataTypeTok{header =} \OtherTok{TRUE}\NormalTok{, }\DataTypeTok{sep =} \StringTok{"|"}\NormalTok{, }\DataTypeTok{dec =} \StringTok{"."}\NormalTok{)}

\CommentTok{#remove unwanted column}
\NormalTok{personality =}\StringTok{ }\KeywordTok{subset}\NormalTok{(personality, }\DataTypeTok{select =} \OperatorTok{-}\KeywordTok{c}\NormalTok{(V00))}

\CommentTok{#create two columns from date_test, one year and one week}
\NormalTok{temp =}\StringTok{ }\KeywordTok{strsplit}\NormalTok{(}\KeywordTok{as.character}\NormalTok{(personality}\OperatorTok{$}\NormalTok{date_test), }\StringTok{" "}\NormalTok{)}
\NormalTok{personality}\OperatorTok{$}\NormalTok{date =}\StringTok{ }\KeywordTok{matrix}\NormalTok{(}\KeywordTok{unlist}\NormalTok{(temp), }\DataTypeTok{ncol=}\DecValTok{2}\NormalTok{, }\DataTypeTok{byrow=}\OtherTok{TRUE}\NormalTok{)[,}\DecValTok{1}\NormalTok{]}
\NormalTok{personality}\OperatorTok{$}\NormalTok{year =}\StringTok{ }\KeywordTok{format}\NormalTok{(}\KeywordTok{as.Date}\NormalTok{(personality}\OperatorTok{$}\NormalTok{date, }\StringTok{"%m/%d/%Y"}\NormalTok{), }\DataTypeTok{format=}\StringTok{"%Y"}\NormalTok{)}
\NormalTok{personality}\OperatorTok{$}\NormalTok{week =}\StringTok{ }\KeywordTok{format}\NormalTok{(}\KeywordTok{as.Date}\NormalTok{(personality}\OperatorTok{$}\NormalTok{date, }\StringTok{"%m/%d/%Y"}\NormalTok{), }\DataTypeTok{format=} \StringTok{"%W"}\NormalTok{)}
\NormalTok{personality =}\StringTok{ }\KeywordTok{subset}\NormalTok{(personality, }\DataTypeTok{select =} \OperatorTok{-}\KeywordTok{c}\NormalTok{(date, date_test))}
\KeywordTok{library}\NormalTok{(dplyr)}
\end{Highlighting}
\end{Shaded}

\begin{verbatim}
## Warning: package 'dplyr' was built under R version 3.5.3
\end{verbatim}

\begin{Shaded}
\begin{Highlighting}[]
\NormalTok{personality =}\StringTok{ }\NormalTok{personality  }\OperatorTok\StringTok{ }\KeywordTok{select}\NormalTok{(md5_email, year, week, }\KeywordTok{everything}\NormalTok{())}

\CommentTok{#Sort the new data frame by YEAR, WEEK so the newest tests are at the top of the df}
\NormalTok{personality =}\StringTok{ }\NormalTok{personality[}\KeywordTok{order}\NormalTok{(}\OperatorTok{-}\NormalTok{(}\KeywordTok{as.numeric}\NormalTok{(personality}\OperatorTok{$}\NormalTok{year)), }\OperatorTok{-}\NormalTok{(}\KeywordTok{as.numeric}\NormalTok{(personality}\OperatorTok{$}\NormalTok{week))), ]}

\CommentTok{#remove duplicates using the unique function based on the column "md5_email"}
\NormalTok{unique =}\StringTok{ }\KeywordTok{unique}\NormalTok{(personality}\OperatorTok{$}\NormalTok{md5_email)}
\NormalTok{rows =}\StringTok{ }\KeywordTok{match}\NormalTok{(unique, personality}\OperatorTok{$}\NormalTok{md5_email)}

\NormalTok{unique_personalities =}\StringTok{ }\NormalTok{personality[rows,]}

\CommentTok{#Save the data frame in the "pipe-delimited format" as "personality-clean.txt"}
\KeywordTok{write.table}\NormalTok{(unique_personalities,}\StringTok{"personality-clean.txt"}\NormalTok{,}\DataTypeTok{sep=}\StringTok{"|"}\NormalTok{, }\DataTypeTok{row.names=}\OtherTok{FALSE}\NormalTok{)}

\KeywordTok{cat}\NormalTok{(}\StringTok{"The raw dataset contains "}\NormalTok{,}\KeywordTok{nrow}\NormalTok{(personality), }
\StringTok{" records, and the cleaned dataset has "}\NormalTok{, }\KeywordTok{nrow}\NormalTok{(unique_personalities),}\StringTok{"."}\NormalTok{, }\DataTypeTok{sep =} \StringTok{""}\NormalTok{)}
\end{Highlighting}
\end{Shaded}

\begin{verbatim}
## The raw dataset contains 838 records, and the cleaned dataset has 678.
\end{verbatim}

\hypertarget{variance-and-z-scores}{%
\section{Variance and Z-scores}\label{variance-and-z-scores}}

Write functions for doSummary and sampleVariance and doMode \ldots{}
test these functions in your homework on the
``\href{mailto:monte.shaffer@gmail.com}{\nolinkurl{monte.shaffer@gmail.com}}''
record from the clean dataset. Report your findings. For this
``\href{mailto:monte.shaffer@gmail.com}{\nolinkurl{monte.shaffer@gmail.com}}''
record, also create z-scores. Plot(x,y) where x is the raw scores for
``\href{mailto:monte.shaffer@gmail.com}{\nolinkurl{monte.shaffer@gmail.com}}''
and y is the z-scores from those raw scores. Include the plot in your
assignment, and write 2 sentences describing what pattern you are seeing
and why this pattern is present.

\begin{Shaded}
\begin{Highlighting}[]
\NormalTok{x.norm =}\StringTok{ }\KeywordTok{rnorm}\NormalTok{(}\DecValTok{100}\NormalTok{,}\DecValTok{0}\NormalTok{,}\DecValTok{1}\NormalTok{);}
\NormalTok{s.norm =}\StringTok{ }\KeywordTok{doStatsSummary}\NormalTok{ ( x.norm );}
\KeywordTok{str}\NormalTok{(s.norm);  }\CommentTok{# mode is pretty meaningless on this data}
\end{Highlighting}
\end{Shaded}

\begin{verbatim}
## List of 32
##  $ length           : int 100
##  $ length.na        : int 0
##  $ length.good      : int 100
##  $ mean             : num 0.0341
##  $ mean.trim.05     : num 0.0387
##  $ mean.trim.20     : num 0.0294
##  $ median           : num 0.0888
##  $ MAD              : num 0.902
##  $ IQR              : num 1.19
##  $ quartiles        : Named num [1:3] -0.6032 0.0888 0.5894
##   ..- attr(*, "names")= chr [1:3] "25%" "50%" "75%"
##  $ deciles          : Named num [1:9] -0.9781 -0.6744 -0.5258 -0.1857 0.0888 ...
##   ..- attr(*, "names")= chr [1:9] "10%" "20%" "30%" "40%" ...
##  $ centiles         : Named num [1:99] -1.71 -1.64 -1.45 -1.39 -1.39 ...
##   ..- attr(*, "names")= chr [1:99] "1%" "2%" "3%" "4%" ...
##  $ median.weighted  : num 0.902
##  $ MAD.weighted     : num 0.0888
##  $ max              : num 1.89
##  $ min              : num -2.09
##  $ range            : num 3.98
##  $ xlim             : num [1:2] -2.09 1.89
##  $ max.idx          : num 98
##  $ min.idx          : num 37
##  $ freq.max         : num [1:100] -2.09 -1.71 -1.64 -1.45 -1.39 ...
##  $ mode             : num [1:100] -2.09 -1.71 -1.64 -1.45 -1.39 ...
##  $ which.min.freq   : num [1:100] -2.09 -1.71 -1.64 -1.45 -1.39 ...
##  $ ylim             : int [1:2] 1 1
##  $ sd               : num 0.843
##  $ var              : num 0.71
##  $ var.naive        :List of 3
##   ..$ x.bar: num 0.0341
##   ..$ s.var: num 0.71
##   ..$ s.sd : num 0.843
##  $ var.2step        :List of 3
##   ..$ x.bar: num 0.0341
##   ..$ s.var: num 0.71
##   ..$ s.sd : num 0.843
##  $ shapiro          :List of 4
##   ..$ statistic: Named num 0.991
##   .. ..- attr(*, "names")= chr "W"
##   ..$ p.value  : num 0.78
##   ..$ method   : chr "Shapiro-Wilk normality test"
##   ..$ data.name: chr "xx"
##   ..- attr(*, "class")= chr "htest"
##  $ shapiro.is.normal:List of 3
##   ..$ 0.10: logi TRUE
##   ..$ 0.05: logi TRUE
##   ..$ 0.01: logi TRUE
##  $ outliers.z       :'data.frame':   0 obs. of  2 variables:
##   ..$ value    : Factor w/ 0 levels: 
##   ..$ direction: Factor w/ 0 levels: 
##  $ outliers.IQR     :'data.frame':   0 obs. of  3 variables:
##   ..$ value    : Factor w/ 0 levels: 
##   ..$ fence    : Factor w/ 0 levels: 
##   ..$ direction: Factor w/ 0 levels:
\end{verbatim}

\begin{Shaded}
\begin{Highlighting}[]
\NormalTok{x.unif =}\StringTok{ }\KeywordTok{runif}\NormalTok{(}\DecValTok{100}\NormalTok{,}\DecValTok{0}\NormalTok{,}\DecValTok{1}\NormalTok{);}
\NormalTok{s.unif =}\StringTok{ }\KeywordTok{doStatsSummary}\NormalTok{ ( x.unif );}
\KeywordTok{str}\NormalTok{(s.unif);  }\CommentTok{# mode is pretty meaningless on this data}
\end{Highlighting}
\end{Shaded}

\begin{verbatim}
## List of 32
##  $ length           : int 100
##  $ length.na        : int 0
##  $ length.good      : int 100
##  $ mean             : num 0.476
##  $ mean.trim.05     : num 0.472
##  $ mean.trim.20     : num 0.479
##  $ median           : num 0.481
##  $ MAD              : num 0.385
##  $ IQR              : num 0.52
##  $ quartiles        : Named num [1:3] 0.208 0.481 0.729
##   ..- attr(*, "names")= chr [1:3] "25%" "50%" "75%"
##  $ deciles          : Named num [1:9] 0.105 0.161 0.27 0.373 0.481 ...
##   ..- attr(*, "names")= chr [1:9] "10%" "20%" "30%" "40%" ...
##  $ centiles         : Named num [1:99] 0.0221 0.0356 0.0552 0.0702 0.0798 ...
##   ..- attr(*, "names")= chr [1:99] "1%" "2%" "3%" "4%" ...
##  $ median.weighted  : num 0.385
##  $ MAD.weighted     : num 0.481
##  $ max              : num 0.998
##  $ min              : num 0.00474
##  $ range            : num 0.994
##  $ xlim             : num [1:2] 0.00474 0.99836
##  $ max.idx          : num 78
##  $ min.idx          : num 48
##  $ freq.max         : num [1:100] 0.00474 0.02227 0.03587 0.05577 0.07084 ...
##  $ mode             : num [1:100] 0.00474 0.02227 0.03587 0.05577 0.07084 ...
##  $ which.min.freq   : num [1:100] 0.00474 0.02227 0.03587 0.05577 0.07084 ...
##  $ ylim             : int [1:2] 1 1
##  $ sd               : num 0.281
##  $ var              : num 0.0788
##  $ var.naive        :List of 3
##   ..$ x.bar: num 0.476
##   ..$ s.var: num 0.0788
##   ..$ s.sd : num 0.281
##  $ var.2step        :List of 3
##   ..$ x.bar: num 0.476
##   ..$ s.var: num 0.0788
##   ..$ s.sd : num 0.281
##  $ shapiro          :List of 4
##   ..$ statistic: Named num 0.944
##   .. ..- attr(*, "names")= chr "W"
##   ..$ p.value  : num 0.000348
##   ..$ method   : chr "Shapiro-Wilk normality test"
##   ..$ data.name: chr "xx"
##   ..- attr(*, "class")= chr "htest"
##  $ shapiro.is.normal:List of 3
##   ..$ 0.10: logi FALSE
##   ..$ 0.05: logi FALSE
##   ..$ 0.01: logi FALSE
##  $ outliers.z       :'data.frame':   0 obs. of  2 variables:
##   ..$ value    : Factor w/ 0 levels: 
##   ..$ direction: Factor w/ 0 levels: 
##  $ outliers.IQR     :'data.frame':   0 obs. of  3 variables:
##   ..$ value    : Factor w/ 0 levels: 
##   ..$ fence    : Factor w/ 0 levels: 
##   ..$ direction: Factor w/ 0 levels:
\end{verbatim}

\begin{Shaded}
\begin{Highlighting}[]
\NormalTok{v2.norm =}\StringTok{ }\KeywordTok{doSampleVariance}\NormalTok{(x.norm, }\StringTok{"two-pass"}\NormalTok{);}
\NormalTok{v2b.norm =}\StringTok{ }\KeywordTok{doSampleVariance}\NormalTok{(x.norm);  }\CommentTok{# default value is "two-pass" in the function}
\NormalTok{v2c.norm =}\StringTok{ }\KeywordTok{doSampleVariance}\NormalTok{(x.norm, }\StringTok{"garblideljd=-gook"}\NormalTok{); }\CommentTok{# if logic defaults to "two-pass"}

\KeywordTok{unlist}\NormalTok{(v2.norm);}
\end{Highlighting}
\end{Shaded}

\begin{verbatim}
##          sumXs sumDiffSquared       variance 
##      3.4073233     70.3247003      0.7103505
\end{verbatim}

\begin{Shaded}
\begin{Highlighting}[]
\KeywordTok{unlist}\NormalTok{(v2b.norm);}
\end{Highlighting}
\end{Shaded}

\begin{verbatim}
##          sumXs sumDiffSquared       variance 
##      3.4073233     70.3247003      0.7103505
\end{verbatim}

\begin{Shaded}
\begin{Highlighting}[]
\KeywordTok{unlist}\NormalTok{(v2c.norm);}
\end{Highlighting}
\end{Shaded}

\begin{verbatim}
##          sumXs sumDiffSquared       variance 
##      3.4073233     70.3247003      0.7103505
\end{verbatim}

\hypertarget{z-scores}{%
\subsection{Z-Scores}\label{z-scores}}

Application of z-score

\begin{Shaded}
\begin{Highlighting}[]
\CommentTok{#library(digest);}
\CommentTok{#md5_monte = digest("monte.shaffer@gmail.com", algo="md5");  # no workee???}
\NormalTok{md5_monte =}\StringTok{ "b62c73cdaf59e0a13de495b84030734e"}\NormalTok{;}
\CommentTok{#get the Monte row from the clean dataset}
\NormalTok{monte =}\StringTok{ }\NormalTok{unique_personalities[unique_personalities}\OperatorTok{$}\NormalTok{md5_email }\OperatorTok{==}\StringTok{ }\NormalTok{md5_monte, ]}
\NormalTok{monte =}\StringTok{ }\NormalTok{monte[}\KeywordTok{c}\NormalTok{(}\OperatorTok{-}\DecValTok{1}\NormalTok{, }\DecValTok{-2}\NormalTok{, }\DecValTok{-3}\NormalTok{)]}

\NormalTok{data =}\StringTok{ }\KeywordTok{zScores}\NormalTok{(monte)}
\KeywordTok{plot}\NormalTok{(}\KeywordTok{unlist}\NormalTok{(data[}\StringTok{"value"}\NormalTok{, ]), }\KeywordTok{unlist}\NormalTok{(data[}\StringTok{"z-score"}\NormalTok{, ]), }\DataTypeTok{ylab =} \StringTok{"z-score"}\NormalTok{, }
     \DataTypeTok{xlab =} \StringTok{"raw value"}\NormalTok{, }\DataTypeTok{main =} \StringTok{"monte.shaffer@gmail.com"}\NormalTok{, }\DataTypeTok{xlim =}\NormalTok{ (}\KeywordTok{c}\NormalTok{(}\DecValTok{1}\NormalTok{,}\DecValTok{6}\NormalTok{)), }
     \DataTypeTok{ylim =}\NormalTok{ (}\KeywordTok{c}\NormalTok{(}\OperatorTok{-}\DecValTok{4}\NormalTok{, }\DecValTok{4}\NormalTok{)))}
\KeywordTok{abline}\NormalTok{(}\DataTypeTok{v =} \KeywordTok{rowMeans}\NormalTok{(data[}\StringTok{"value"}\NormalTok{, ]), }\DataTypeTok{lty =} \DecValTok{3}\NormalTok{, }\DataTypeTok{col =} \StringTok{"blue"}\NormalTok{)}
\end{Highlighting}
\end{Shaded}

\includegraphics{04_datasets_notebook_resubmission_files/figure-latex/mychunk-apply-z-score-1.pdf}

\begin{Shaded}
\begin{Highlighting}[]
\KeywordTok{writeLines}\NormalTok{(}\StringTok{"The zscore is a measure of how far from the mean a data point is, to phrase it }
\StringTok{informally. }
\StringTok{           }\CharTok{\textbackslash{}n}\StringTok{ The plot of the monte sample shows that as the raw value gets higher, the z-score }
\StringTok{           gets lower. The }\CharTok{\textbackslash{}n}\StringTok{ mean of the sample is show by the blue dashed line at 3.84, so }
\StringTok{           the z-scores closest to zero should }\CharTok{\textbackslash{}n}\StringTok{ be associated with raw values near the mean."}\NormalTok{)}
\end{Highlighting}
\end{Shaded}

\begin{verbatim}
## The zscore is a measure of how far from the mean a data point is, to phrase it 
## informally. 
##            
##  The plot of the monte sample shows that as the raw value gets higher, the z-score 
##            gets lower. The 
##  mean of the sample is show by the blue dashed line at 3.84, so 
##            the z-scores closest to zero should 
##  be associated with raw values near the mean.
\end{verbatim}

\hypertarget{will-vs-denzel}{%
\section{Will vs Denzel}\label{will-vs-denzel}}

\begin{Shaded}
\begin{Highlighting}[]
\KeywordTok{source}\NormalTok{( }\KeywordTok{paste0}\NormalTok{(local.path,}\StringTok{"/functions/functions-imdb.R"}\NormalTok{), }\DataTypeTok{local=}\NormalTok{T );}

\NormalTok{nmid =}\StringTok{ "nm0000226"}\NormalTok{;}
\NormalTok{will =}\StringTok{ }\KeywordTok{grabFilmsForPerson}\NormalTok{(nmid);}

\NormalTok{nmid =}\StringTok{ "nm0000243"}\NormalTok{;}
\NormalTok{denzel =}\StringTok{ }\KeywordTok{grabFilmsForPerson}\NormalTok{(nmid);}

\CommentTok{#}
\end{Highlighting}
\end{Shaded}

Compare Will Smith and Denzel Washington. {[}See 03\_n greater 1-v2.txt
for the necessary functions and will-vs-denzel.txt for some sample code
and in DROPBOX:
\_\_student\_access\_\_\unit\_01\_exploratory\_data\_analysis\week\_02\imdb-example
{]} You will have to create a new variable \$millions.2000 that converts
each movie's \$millions based on the \$year of the movie, so all dollars
are in the same time frame. You will need inflation data from about
1980-2020 to make this work.

\hypertarget{boxplot-of-top-50-movies-using-raw-dollars}{%
\subsection{BoxPlot of Top-50 movies using Raw
Dollars}\label{boxplot-of-top-50-movies-using-raw-dollars}}

\begin{Shaded}
\begin{Highlighting}[]
\KeywordTok{par}\NormalTok{(}\DataTypeTok{mfrow=}\KeywordTok{c}\NormalTok{(}\DecValTok{1}\NormalTok{,}\DecValTok{2}\NormalTok{));}
    \KeywordTok{boxplot}\NormalTok{(will}\OperatorTok{$}\NormalTok{movies}\FloatTok{.50}\OperatorTok{$}\NormalTok{millions, }\DataTypeTok{main=}\NormalTok{will}\OperatorTok{$}\NormalTok{name, }\DataTypeTok{ylim=}\KeywordTok{c}\NormalTok{(}\DecValTok{0}\NormalTok{,}\DecValTok{360}\NormalTok{), }\DataTypeTok{ylab=}\StringTok{"Raw Millions"}\NormalTok{ );}
    \KeywordTok{boxplot}\NormalTok{(denzel}\OperatorTok{$}\NormalTok{movies}\FloatTok{.50}\OperatorTok{$}\NormalTok{millions, }\DataTypeTok{main=}\NormalTok{denzel}\OperatorTok{$}\NormalTok{name, }\DataTypeTok{ylim=}\KeywordTok{c}\NormalTok{(}\DecValTok{0}\NormalTok{,}\DecValTok{360}\NormalTok{), }\DataTypeTok{ylab=}\StringTok{"Raw Millions"}\NormalTok{ );}
\end{Highlighting}
\end{Shaded}

\includegraphics{04_datasets_notebook_resubmission_files/figure-latex/mychunk-boxplot-raw-1.pdf}

\begin{Shaded}
\begin{Highlighting}[]
    \KeywordTok{par}\NormalTok{(}\DataTypeTok{mfrow=}\KeywordTok{c}\NormalTok{(}\DecValTok{1}\NormalTok{,}\DecValTok{1}\NormalTok{));}
\end{Highlighting}
\end{Shaded}

\hypertarget{side-by-side-comparisons}{%
\subsection{Side-by-Side Comparisons}\label{side-by-side-comparisons}}

Build side-by-side box plots on several of the variables (including \#6)
to compare the two movie stars. After each box plot, write 2+ sentence
describing what you are seeing, and what conclusions you can logically
make. You will need to review what the box plot is showing with the box
portion, the divider in the box, and the whiskers.

\hypertarget{adjusted-dollars-2000}{%
\subsubsection{Adjusted Dollars (2000)}\label{adjusted-dollars-2000}}

\begin{Shaded}
\begin{Highlighting}[]
\CommentTok{#inflation: https://inflationdata.com/inflation/Inflation_Articles/CalculateInflation.asp}
\CommentTok{#CPI: https://inflationdata.com/Inflation/Consumer_Price_Index/HistoricalCPI.aspx?reloaded=true#Table }
\NormalTok{cpi =}\StringTok{ }\KeywordTok{read.table}\NormalTok{(}\StringTok{"C:}\CharTok{\textbackslash{}\textbackslash{}}\StringTok{Users}\CharTok{\textbackslash{}\textbackslash{}}\StringTok{jsmit}\CharTok{\textbackslash{}\textbackslash{}}\StringTok{Desktop}\CharTok{\textbackslash{}\textbackslash{}}\StringTok{WSU}\CharTok{\textbackslash{}\textbackslash{}}\StringTok{DataAnalytics}\CharTok{\textbackslash{}\textbackslash{}}\StringTok{STAT419}\CharTok{\textbackslash{}\textbackslash{}}\StringTok{WSU_STATS419_FALL2020}\CharTok{\textbackslash{}\textbackslash{}}\StringTok{datasets}\CharTok{\textbackslash{}\textbackslash{}}\StringTok{CPI.csv"}\NormalTok{, }\DataTypeTok{header =} \OtherTok{TRUE}\NormalTok{, }\DataTypeTok{sep =} \StringTok{","}\NormalTok{)}

\NormalTok{mean =}\StringTok{ }\KeywordTok{rowMeans}\NormalTok{(cpi[ , }\DecValTok{-1}\NormalTok{], }\DataTypeTok{na.rm=} \OtherTok{TRUE}\NormalTok{)}
\NormalTok{CPI =}\StringTok{ }\KeywordTok{as.data.frame}\NormalTok{(}\KeywordTok{cbind}\NormalTok{(}\DataTypeTok{year =}\NormalTok{ cpi}\OperatorTok{$}\NormalTok{AR ,}\DataTypeTok{cpi =}\NormalTok{ mean))}
\NormalTok{cpi2000 =}\StringTok{ }\NormalTok{CPI[}\DecValTok{1}\NormalTok{,}\DecValTok{2}\NormalTok{]}

\KeywordTok{library}\NormalTok{(dplyr)}
\NormalTok{D =}\StringTok{ }\NormalTok{denzel}\OperatorTok{$}\NormalTok{movies}\FloatTok{.50} \OperatorTok\StringTok{ }\KeywordTok{inner_join}\NormalTok{(CPI, }\DataTypeTok{by =} \StringTok{"year"}\NormalTok{)}
\NormalTok{W =}\StringTok{ }\NormalTok{will}\OperatorTok{$}\NormalTok{movies}\FloatTok{.50} \OperatorTok\StringTok{ }\KeywordTok{inner_join}\NormalTok{(CPI, }\DataTypeTok{by =} \StringTok{"year"}\NormalTok{)}

\NormalTok{D}\OperatorTok{$}\NormalTok{millions_}\DecValTok{2000}\NormalTok{ =}\StringTok{ }\NormalTok{(D}\OperatorTok{$}\NormalTok{millions}\OperatorTok{*}\NormalTok{cpi2000)}\OperatorTok{/}\NormalTok{D}\OperatorTok{$}\NormalTok{cpi}
\NormalTok{W}\OperatorTok{$}\NormalTok{millions_}\DecValTok{2000}\NormalTok{ =}\StringTok{ }\NormalTok{(W}\OperatorTok{$}\NormalTok{millions}\OperatorTok{*}\NormalTok{cpi2000)}\OperatorTok{/}\NormalTok{W}\OperatorTok{$}\NormalTok{cpi}
\end{Highlighting}
\end{Shaded}

\begin{Shaded}
\begin{Highlighting}[]
\KeywordTok{par}\NormalTok{(}\DataTypeTok{mfrow=}\KeywordTok{c}\NormalTok{(}\DecValTok{1}\NormalTok{,}\DecValTok{2}\NormalTok{));}
\KeywordTok{boxplot}\NormalTok{(W}\OperatorTok{$}\NormalTok{millions_}\DecValTok{2000}\NormalTok{, }\DataTypeTok{main=}\NormalTok{will}\OperatorTok{$}\NormalTok{name, }\DataTypeTok{ylim=}\KeywordTok{c}\NormalTok{(}\DecValTok{0}\NormalTok{,}\DecValTok{550}\NormalTok{), }\DataTypeTok{ylab=}\StringTok{"Millions in the year 2000"}\NormalTok{)}
\KeywordTok{boxplot}\NormalTok{(D}\OperatorTok{$}\NormalTok{millions_}\DecValTok{2000}\NormalTok{, }\DataTypeTok{main=}\NormalTok{denzel}\OperatorTok{$}\NormalTok{name, }\DataTypeTok{ylim=}\KeywordTok{c}\NormalTok{(}\DecValTok{0}\NormalTok{,}\DecValTok{550}\NormalTok{), }\DataTypeTok{ylab=}\StringTok{"Millions in the year 2000"}\NormalTok{)}
\end{Highlighting}
\end{Shaded}

\includegraphics{04_datasets_notebook_resubmission_files/figure-latex/mychunk-side-by-side-1.pdf}

\begin{Shaded}
\begin{Highlighting}[]
\KeywordTok{writeLines}\NormalTok{(}\StringTok{"The boxplots of box office sales show that the median for the two stars is about }
\StringTok{the same.}\CharTok{\textbackslash{}n}\StringTok{ There is greater variation in the amount that Will Smith movies earn as shown by }
\StringTok{the longer }\CharTok{\textbackslash{}n}\StringTok{ whiskers and larger interquartile range, while Denzel's movies appear to be }
\StringTok{more consistent."}\NormalTok{)}
\end{Highlighting}
\end{Shaded}

\begin{verbatim}
## The boxplots of box office sales show that the median for the two stars is about 
## the same.
##  There is greater variation in the amount that Will Smith movies earn as shown by 
## the longer 
##  whiskers and larger interquartile range, while Denzel's movies appear to be 
## more consistent.
\end{verbatim}

\begin{Shaded}
\begin{Highlighting}[]
\KeywordTok{par}\NormalTok{(}\DataTypeTok{mfrow=}\KeywordTok{c}\NormalTok{(}\DecValTok{1}\NormalTok{,}\DecValTok{2}\NormalTok{));}
\KeywordTok{boxplot}\NormalTok{(W}\OperatorTok{$}\NormalTok{ratings, }\DataTypeTok{main=}\NormalTok{will}\OperatorTok{$}\NormalTok{name, }\DataTypeTok{ylim=}\KeywordTok{c}\NormalTok{(}\DecValTok{0}\NormalTok{,}\DecValTok{10}\NormalTok{), }\DataTypeTok{ylab=}\StringTok{"Movie Ratings"}\NormalTok{)}
\KeywordTok{boxplot}\NormalTok{(D}\OperatorTok{$}\NormalTok{ratings, }\DataTypeTok{main=}\NormalTok{denzel}\OperatorTok{$}\NormalTok{name, }\DataTypeTok{ylim=}\KeywordTok{c}\NormalTok{(}\DecValTok{0}\NormalTok{,}\DecValTok{10}\NormalTok{), }\DataTypeTok{ylab=}\StringTok{"Movie Ratings"}\NormalTok{)}
\end{Highlighting}
\end{Shaded}

\includegraphics{04_datasets_notebook_resubmission_files/figure-latex/mychunk-side-by-side-2.pdf}

\begin{Shaded}
\begin{Highlighting}[]
\KeywordTok{writeLines}\NormalTok{(}\StringTok{"The boxplots of movie ratings show that the median for the two stars is again }
\StringTok{very close,}\CharTok{\textbackslash{}n}\StringTok{ but Denzel's score is a little higher.  There is a larger range of ratings }
\StringTok{for Will Smith's movies,}\CharTok{\textbackslash{}n}\StringTok{ shown by the longer whiskers. The interquartile ranges have }
\StringTok{similar sizes for both actors"}\NormalTok{)}
\end{Highlighting}
\end{Shaded}

\begin{verbatim}
## The boxplots of movie ratings show that the median for the two stars is again 
## very close,
##  but Denzel's score is a little higher.  There is a larger range of ratings 
## for Will Smith's movies,
##  shown by the longer whiskers. The interquartile ranges have 
## similar sizes for both actors
\end{verbatim}

\begin{Shaded}
\begin{Highlighting}[]
\KeywordTok{par}\NormalTok{(}\DataTypeTok{mfrow=}\KeywordTok{c}\NormalTok{(}\DecValTok{1}\NormalTok{,}\DecValTok{2}\NormalTok{));}
\KeywordTok{boxplot}\NormalTok{(W}\OperatorTok{$}\NormalTok{metacritic, }\DataTypeTok{main=}\NormalTok{will}\OperatorTok{$}\NormalTok{name, }\DataTypeTok{ylim=}\KeywordTok{c}\NormalTok{(}\DecValTok{0}\NormalTok{,}\DecValTok{100}\NormalTok{), }\DataTypeTok{ylab=}\StringTok{"Metacritic Scores"}\NormalTok{)}
\KeywordTok{boxplot}\NormalTok{(D}\OperatorTok{$}\NormalTok{metacritic, }\DataTypeTok{main=}\NormalTok{denzel}\OperatorTok{$}\NormalTok{name, }\DataTypeTok{ylim=}\KeywordTok{c}\NormalTok{(}\DecValTok{0}\NormalTok{,}\DecValTok{100}\NormalTok{), }\DataTypeTok{ylab=}\StringTok{"Metacritic Scores"}\NormalTok{)}
\end{Highlighting}
\end{Shaded}

\includegraphics{04_datasets_notebook_resubmission_files/figure-latex/mychunk-side-by-side-3.pdf}

\begin{Shaded}
\begin{Highlighting}[]
\KeywordTok{writeLines}\NormalTok{(}\StringTok{"The metacritic scores for the two actors show the mean score for Denzel to }
\StringTok{be slightly higher.}\CharTok{\textbackslash{}n}\StringTok{ The whisker legth is similar, as is the size of the interquartile }
\StringTok{range."}\NormalTok{)}
\end{Highlighting}
\end{Shaded}

\begin{verbatim}
## The metacritic scores for the two actors show the mean score for Denzel to 
## be slightly higher.
##  The whisker legth is similar, as is the size of the interquartile 
## range.
\end{verbatim}

\end{document}
