\documentclass[]{article}
\usepackage[left=1in,top=1in,right=1in,bottom=1in]{geometry}


%%%% more monte %%%%
% thispagestyle{empty}
% https://stackoverflow.com/questions/2166557/how-to-hide-the-page-number-in-latex-on-first-page-of-a-chapter
\usepackage{color}
% \usepackage[table]{xcolor} % are they using color?

% \definecolor{WSU.crimson}{HTML}{981e32}
% \definecolor{WSU.gray}{HTML}{5e6a71}

% \definecolor{shadecolor}{RGB}{248,248,248}
\definecolor{WSU.crimson}{RGB}{152,30,50} % use http://colors.mshaffer.com to convert from 981e32
\definecolor{WSU.gray}{RGB}{94,106,113}

%%%%%%%%%%%%%%%%%%%%%%%%%%%%

\newcommand*{\authorfont}{\fontfamily{phv}\selectfont}
\usepackage{lmodern}


  \usepackage[T1]{fontenc}
  \usepackage[utf8]{inputenc}




\usepackage{abstract}
\renewcommand{\abstractname}{}    % clear the title
\renewcommand{\absnamepos}{empty} % originally center

\renewenvironment{abstract}
 {{%
    \setlength{\leftmargin}{0mm}
    \setlength{\rightmargin}{\leftmargin}%
  }%
  \relax}
 {\endlist}

\makeatletter
\def\@maketitle{%
  \pagestyle{empty}
  \newpage
%  \null
%  \vskip 2em%
%  \begin{center}%
  \let \footnote \thanks
    {\fontsize{18}{20}\selectfont\raggedright  \setlength{\parindent}{0pt} \@title \par}%
}
%\fi
\makeatother









\title{\textbf{\textcolor{WSU.crimson}{Will Smith Versus Denzel
Washington}} \newline \textbf{\textcolor{WSU.gray}{Which is the better
actor?}}  }
 

%  

% \author{ \Large true \hfill \normalsize \emph{} }
\author{\Large Jessica
Smith\vspace{0.05in} \newline\normalsize\emph{Washington State
University}  }


\date{December 08, 2020}
\setcounter{secnumdepth}{3}

\usepackage{titlesec}
% See the link above: KOMA classes are not compatible with titlesec any more. Sorry.
% https://github.com/jbezos/titlesec/issues/11
\titleformat*{\section}{\bfseries}
\titleformat*{\subsection}{\bfseries\itshape}
\titleformat*{\subsubsection}{\itshape}
\titleformat*{\paragraph}{\itshape}
\titleformat*{\subparagraph}{\itshape}

% https://code.usgs.gov/usgs/norock/irvine_k/ip-092225/


%\titleformat*{\section}{\normalsize\bfseries}
%\titleformat*{\subsection}{\normalsize\itshape}
%\titleformat*{\subsubsection}{\normalsize\itshape}
%\titleformat*{\paragraph}{\normalsize\itshape}
%\titleformat*{\subparagraph}{\normalsize\itshape}

% https://tex.stackexchange.com/questions/233866/one-column-multicol-environment#233904
\usepackage{environ}
\NewEnviron{auxmulticols}[1]{%
  \ifnum#1<2\relax% Fewer than 2 columns
    %\vspace{-\baselineskip}% Possible vertical correction
    \BODY
  \else% More than 1 column
    \begin{multicols}{#1}
      \BODY
    \end{multicols}%
  \fi
}





\usepackage{natbib}
\setcitestyle{aysep={}} %% no year, comma just year
% \usepackage[numbers]{natbib}
\bibliographystyle{plainnat}



\usepackage[strings]{underscore} % protect underscores in most circumstances




\newtheorem{hypothesis}{Hypothesis}
\usepackage{setspace}


%%%%%%%%%%%%%%%%%%%%%%%%%%%%%%%%%%%%%%%%%%%%%%%%%%%%%
%%% MONTE ADDS %%%

\usepackage{fancyhdr} % fancy header 
\usepackage{lastpage} % last page 

\usepackage{multicol}


\usepackage{etoolbox}
\AtBeginEnvironment{quote}{\singlespacing\small}
% https://tex.stackexchange.com/questions/325695/how-to-style-blockquote


\usepackage{soul}			%% allows strike-through
\usepackage{url}			%% fixes underscores in urls
\usepackage{csquotes}		%% allows \textquote in references
\usepackage{rotating}		%% allows table and box rotation
\usepackage{caption}		%% customize caption information
\usepackage{booktabs}		%% enhance table/tabular environment
\usepackage{tabularx}		%% width attributes updates tabular
\usepackage{enumerate}		%% special item environment
\usepackage{enumitem}		%% special item environment

\usepackage{lineno}		%% allows linenumbers for editing using \linenumbers
\usepackage{hanging}


\usepackage{mathtools}  	%% also loads amsmath
\usepackage{bm}		%% bold-math
\usepackage{scalerel}	%% scale one element (make one beta bigger font)

\newcommand{\gFrac}[2]{ \genfrac{}{}{0pt}{1}{{#1}}{#2} }

\newcommand{\betaSH}[3]{  \gFrac{\text{\tiny #1}}{{\text{\tiny #2}}}\hat{\beta}_{\text{#3}}   }
\newcommand{\betaSB}[3]{              ^{\text{#1}} _{\text{#2}} \bm{\beta} _{\text{#3}}                   }  %% bold
\newcommand{\bigEQ}{  \scaleobj{1.5}{{\ }= } }
\newcommand{\bigP}[1]{  \scaleobj{1.5}{#1 } }





\usepackage{endnotes}  % he already does this ...
\renewcommand{\enotesize}{\normalsize}
% https://tex.stackexchange.com/questions/99984/endnotes-do-not-be-superscript-and-add-a-space
\renewcommand\makeenmark{\textsuperscript{[\theenmark]}} % in brackets %
% https://tex.stackexchange.com/questions/31574/how-to-control-the-indent-in-endnotes
\patchcmd{\enoteformat}{1.8em}{0pt}{}{}

\patchcmd{\theendnotes}
  {\makeatletter}
  {\makeatletter\renewcommand\makeenmark{\textbf{[\theenmark]} }}
  {}{}



% https://tex.stackexchange.com/questions/141906/configuring-footnote-position-and-spacing

\addtolength{\footnotesep}{5mm} % change to 1mm

\renewcommand{\thefootnote}{\textbf{\arabic{footnote}}}
\let\footnote=\endnote
%\renewcommand*{\theendnote}{\alph{endnote}}
%\renewcommand{\theendnote}{\textbf{\arabic{endnote}}}


\renewcommand*{\notesname}{}

\makeatletter
\def\enoteheading{\section*{\notesname
  \@mkboth{\MakeUppercase{\notesname}}{\MakeUppercase{\notesname}}}%
  \mbox{}\par\vskip-2.3\baselineskip\noindent\rule{.5\textwidth}{0.4pt}\par\vskip\baselineskip}
\makeatother


\renewcommand*{\contentsname}{TABLE OF CONTENTS}

\renewcommand*{\refname}{REFERENCES}


%\usepackage{subfigure}
\usepackage{subcaption}

\captionsetup{labelfont=bf}  % Make Table / Figure bold

%%% you could add elements here ... monte says .... %%%
%\usepackage{mypackageForCapitalH}


%%%%%%%%%%%%%%%%%%%%%%%%%%%%%%%%%%%%%%%%%%%%%%%%%%%%%

% set default figure placement to htbp
\makeatletter
\def\fps@figure{htbp}
\makeatother


% move the hyperref stuff down here, after header-includes, to allow for - \usepackage{hyperref}

\makeatletter
\@ifpackageloaded{hyperref}{}{%
\ifxetex
  \PassOptionsToPackage{hyphens}{url}\usepackage[setpagesize=false, % page size defined by xetex
              unicode=false, % unicode breaks when used with xetex
              xetex]{hyperref}
\else
  \PassOptionsToPackage{hyphens}{url}\usepackage[draft,unicode=true]{hyperref}
\fi
}

\@ifpackageloaded{color}{
    \PassOptionsToPackage{usenames,dvipsnames}{color}
}{%
    \usepackage[usenames,dvipsnames]{color}
}
\makeatother
\hypersetup{breaklinks=true,
            bookmarks=true,
            pdfauthor={Jessica Smith (Washington State University)},
             pdfkeywords = {Will Smith, Denzel Washington, IMDB,
Multivariate Analysis},  
            pdftitle={Will Smith Versus Denzel Washington: Which is the
better actor?},
            colorlinks=true,
            citecolor=blue,
            urlcolor=blue,
            linkcolor=magenta,
            pdfborder={0 0 0}}
\urlstyle{same}  % don't use monospace font for urls

% Add an option for endnotes. -----

%
% add tightlist ----------
\providecommand{\tightlist}{%
\setlength{\itemsep}{0pt}\setlength{\parskip}{0pt}}

% add some other packages ----------

% \usepackage{multicol}
% This should regulate where figures float
% See: https://tex.stackexchange.com/questions/2275/keeping-tables-figures-close-to-where-they-are-mentioned
\usepackage[section]{placeins}



\pagestyle{fancy}   
\lhead{\textcolor{WSU.crimson}{\textbf{ Will Smith Versus Denzel
Washington }}}
\chead{}
\rhead{\textcolor{WSU.gray}{\textbf{  Page\ \thepage\ of\ \protect\pageref{LastPage} }}}
\lfoot{}
\cfoot{}
\rfoot{}


\begin{document}
	
% \pagenumbering{arabic}% resets `page` counter to 1 
%    

% \maketitle

{% \usefont{T1}{pnc}{m}{n}
\setlength{\parindent}{0pt}
\thispagestyle{plain}
{\fontsize{18}{20}\selectfont\raggedright 
\maketitle  % title \par  

}

{
   \vskip 13.5pt\relax \normalsize\fontsize{11}{12} 
   
\textbf{\authorfont Jessica Smith} \hskip 15pt \emph{\small Washington
State University}   

}

}








\begin{abstract}

    \hbox{\vrule height .2pt width 39.14pc}

    \vskip 8.5pt % \small 

\noindent This project explored multiple variables from the IMDB
database in order to determine which actor is better - Denzel Washington
or Will Smith. A mathematical analysis was performed to quantitatively
evaluate which actor is better using appropriate data features selected
by the author. An adjacency matrix was constructed from data on 50
contemporary actors, and the eigen ratings of each actor were
calculated. The results of the multivariate analysis showed that Denzel
Washington can be considered the better actor.


\vskip 8.5pt \noindent \textbf{\underline{Keywords}:} Will Smith, Denzel
Washington, IMDB, Multivariate Analysis \par

    




    
    \hbox{\vrule height .2pt width 39.14pc}
    \vskip 5pt 
    \hfill \textbf{\textcolor{WSU.gray}{ December 08, 2020 } }
    \vskip 5pt 
    
\end{abstract}


\vskip -8.5pt



 % removetitleabstract

\noindent  

\section{Introduction}
\label{sec:intro}

The IMDB database is a public repository containing information related
to movies and actors, including cast, crew, ratings, and earnings data.
This project attempted to investigate, compare, and draw conclusions
regarding two well-known actors by performing multivariate analysis on
the data gleaned from the IMDB website. Previous work into the subject
evaluated the actors by considering box office sales and IMDB movie
rankings. It was shown that the median box office sales, when adjusted
for inflation, were about the same for the two stars. The median
Metacritic ratings for each actors films were also similar and did not
enable a meaningful distinction. As the results were inconclusive,
further investigation is needed to provide deterministic insights into
which actor is ``better''. This project will look at multiple variables
and perform a mathematical analysis to attempt to quantitatively
evaluate which actor is better using the available data features. This
an important demonstration of how data analysis can be used to
empirically inform a response to an otherwise hard to answer and
somewhat subjective question. The author has no predisposition toward
either actor, and the information presented is intended to provide an
unbiased and data driven perspective.

\section{Overview}
\label{sec:Overview}

Denzel Washington and Will Smith are two well-known movie stars with
comparable ratings and extensive filmographies. Denzel is 66 years old
and made his first movie in 1981, while Will Smith is 52 and made his
first movies in 1992. Will Smith has 111 films listed on IMDB with a
mean rating of 6.2, while Denzel has 61 films listed with a mean of 6.8.
Both actors are listed in the top 500 on IMDB, and differentiating an
advantage between the two on the basis of published statistics has
proven challenging.\newline

The IMDB data contains metrics on movies in which the actors have
appeared including ratings, box office sales, and Metacritic scores, as
well as info on the cast and crew. For actors, their rating within each
movie is available, allowing for the determination of whether their role
was a lead or whether they played a less prominent character.
Demographic data and star meter rank was also available for each actor.
\newline

The concept of which actor is ``better'' is an inherently subjective
measure, and the choice of which factors to consider in the analysis was
given careful consideration. Calculating a diversity score from an
analysis of the gender breakdown of the cast and crew for each movie was
suggested. However, different genres can have different casting
requirements, and the outcome of any analysis may be more indicative of
which genre an actor prefers than anything else. A preliminary
assessment of the diversity index of Will Smith movies, for example,
showed a wide variation in score across genres. A strong correlation
between gender diversity and acting acumen appears unlikely, and
gender/diversity score was not selected as a component of the analysis.
\newline

Another available factor was the ``star meter'' ranking, a measure of
popularity created by IMDB that is a function of the number of credits a
person has, popularity of the work a star appears in, and traffic to the
celebrity's profile. While this initially seemed promising, after closer
inspection, many of the actors used to build the data set had identical
ratings. This rating also appears as a static snapshot that reflects the
score today, rather than at the time a particular movie was released,
and it's unclear how these ratings might change over time. This metric
was deemed unreliable and non-deterministic, and was not included in the
final analysis. \newline

The analysis required building a dataframe for each actor with the
desired covariates. For each film an actor was associated with, the
movie ratings, Metacritic score, box office sales, and actor rank were
selected. The decision was made to only consider movies where the star
had been a headliner, defined as having an actor rank of 1, 2, or 3 for
a given movie. This was done in an attempt to make sure the covariates
were a mainly a function of the actor being considered, not some one
else who may have been in the movie and had a larger role. The
assumption was made that a ``better'' actor would be appear as the lead
more often than in a supporting role, so the ratio of leading roles to
total movies was also considered. Box office sales were adjusted to 2020
dollars, and the mean of the ratings, scores, and sales for each movie
were calculated. \newline

To obtain a more robust analysis, Will and Denzel should be compared to
more actors than just each other. A pool of 48 contemporaries was drawn
from the actorRank2000.rds provided by the instructor, ensuring only
candidates from the modern era were considered. Actor ID's were randomly
selected and screened for quality. The criteria were at least 10 movies
where the actor ranked 3 or lower, and complete data for ratings,
Metacritic scores, and box office sales. Table 1 shows an example of the
first ten actors in the pool and the final selection of covariates that
were used in the analysis. \newline

% latex table generated in R 4.0.2 by xtable 1.8-4 package
% Tue Dec 08 13:21:19 2020
\begin{table}[ht]
\centering
\begin{tabular}{rlrrrr}
  \toprule
 & actor & lead.ratio & avg.ratings & avg.metacritic & avg.box.office \\ 
  \midrule
1 & Will Smith & 0.26 & 6.57 & 51.12 & 186.15 \\ 
  2 & Denzel Washington & 0.61 & 6.97 & 61.42 & 85.52 \\ 
  3 & Adam Beach & 0.24 & 6.28 & 59.20 & 29.66 \\ 
  4 & Juliane Kohler & 0.41 & 6.59 & 68.67 & 3.49 \\ 
  5 & Emmanuelle Chriqui & 0.32 & 5.08 & 45.00 & 38.88 \\ 
  6 & Helen Hunt & 0.41 & 6.37 & 57.07 & 91.78 \\ 
  7 & Demi Moore & 0.43 & 6.09 & 48.75 & 79.66 \\ 
  8 & Keira Knightley & 0.34 & 6.74 & 58.50 & 66.76 \\ 
  9 & Kristen Wiig & 0.33 & 6.26 & 61.47 & 92.57 \\ 
  10 & Sam Worthington & 0.33 & 6.18 & 42.91 & 161.27 \\ 
   \bottomrule
\end{tabular}
\caption{A sample of the actors and covariates used in the analysis.} 
\end{table}


Using the entire pool of 50 actors, the data was standardized by
dividing each value for a covariate by the maximum value for that
feature found in the pool. The result was a 50 X 4 matrix where each
value was a positive integer between 0 and 1. An adjacency matrix was
calculated where the value of the eigenvector in each row/column is the
rating of the actor evaluated against the other actors in the pool for
that factor. The eigenvector ranking quantified the approximate
importance of each actor in the pool. From these eigen-rankings, an
empirical evaluation of which actor is ``better'', Will or Denzel, was
able to be obtained. \newline

\section{Key Findings}
\label{sec:findings}

% latex table generated in R 4.0.2 by xtable 1.8-4 package
% Tue Dec 08 14:36:38 2020
\begin{table}[ht]
\centering
\begin{tabular}{rr}
  \toprule
 & eigen.rank \\ 
  \midrule
Will Smith & 0.17 \\ 
  Denzel Washington & 0.20 \\ 
   \bottomrule
\end{tabular}
\caption{The eigenvector rankings show Denzel Washington is the better actor.} 
\end{table}


The results of the analysis show that Denzel Washington has a higher
eigen-rank than Will Smith, indicating that given the metrics evaluated,
Denzel can be considered the better actor (.20 to .16). These results
are in agreement with the rankings returned from the instructors
Actor-Actor matrix that showed Denzel Washington scoring slightly higher
than Will Smith (57.11 to 50.79). Given the authors lack of preference
at the outset, the relative `closeness' of the results, and the
alignment with existing research outcomes, the results can be concluded
to be reasonable and acceptable.

\section{Conclusion}
\label{sec:conclusion}

Denzel has a higher eigen-rank than Will Smith when considering the
average ratings of the movies in which he starred, the average box
office sales of those movies, and the ratio of leading to non leading
roles. The results of the multivariate analysis have shown that Denzel
Washington is the better actor.




%% appendices go here!


\newpage
\theendnotes

%%%%%%%%%%%%%%%%%%%%%%%%%%%%%%%%%%%  biblio %%%%%%%%
\newpage
\begin{auxmulticols}{1}
\singlespacing 
%%%%%%%%%%%%%%%%%%%%%%%%%%%%%%%%%%%  biblio %%%%%%%%
\end{auxmulticols}

\newpage
{
\hypersetup{linkcolor=black}
\setcounter{tocdepth}{3}
\tableofcontents
}



\end{document}